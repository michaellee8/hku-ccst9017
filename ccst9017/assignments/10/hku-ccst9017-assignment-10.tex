% --------------------------------------------------------------
% This is all preamble stuff that you don't have to worry about.
% Head down to where it says "Start here"
% --------------------------------------------------------------
 
\documentclass[12pt]{article}
 
    \usepackage[margin=1in]{geometry} 
    \usepackage{amsmath,amsthm,amssymb}
     
    \newcommand{\N}{\mathbb{N}}
    \newcommand{\Z}{\mathbb{Z}}
     
    \newenvironment{theorem}[2][Theorem]{\begin{trivlist}
    \item[\hskip \labelsep {\bfseries #1}\hskip \labelsep {\bfseries #2.}]}{\end{trivlist}}
    \newenvironment{lemma}[2][Lemma]{\begin{trivlist}
    \item[\hskip \labelsep {\bfseries #1}\hskip \labelsep {\bfseries #2.}]}{\end{trivlist}}
    \newenvironment{exercise}[2][Exercise]{\begin{trivlist}
    \item[\hskip \labelsep {\bfseries #1}\hskip \labelsep {\bfseries #2.}]}{\end{trivlist}}
    \newenvironment{reflection}[2][Reflection]{\begin{trivlist}
    \item[\hskip \labelsep {\bfseries #1}\hskip \labelsep {\bfseries #2.}]}{\end{trivlist}}
    \newenvironment{proposition}[2][Proposition]{\begin{trivlist}
    \item[\hskip \labelsep {\bfseries #1}\hskip \labelsep {\bfseries #2.}]}{\end{trivlist}}
    \newenvironment{corollary}[2][Corollary]{\begin{trivlist}
    \item[\hskip \labelsep {\bfseries #1}\hskip \labelsep {\bfseries #2.}]}{\end{trivlist}}
    \newenvironment{question}[2][Question]{\begin{trivlist}
    \item[\hskip \labelsep {\bfseries #1}\hskip \labelsep {\bfseries #2.}]}{\end{trivlist}}
    \newenvironment{answer}[2][Answer]{\begin{trivlist}
    \item[\hskip \labelsep {\bfseries #1}\hskip \labelsep {\bfseries #2.}]}{\end{trivlist}}
    
     
\begin{document}

% --------------------------------------------------------------
%                         Start here
% --------------------------------------------------------------

%\renewcommand{\qedsymbol}{\filledbox}

\title{Assignment 10}%replace X with the appropriate number
\author{Michael Lee\\ %replace with your name
	CCST9017 - Hidden Order in Daily Life: A Mathematical Perspective \\
	University Number 3035569110 \\
	Tutorial Group 009
} %if necessary, replace with your course title


\maketitle

\begin{question}{Q. 1}
	Thoughts on PGP
\end{question}
\begin{answer}{Q. 1}
    \hfill \par
    I am totally agree with his statements. \par
    The first reason is that privacy should be regarded as a basic human right.
    No one should be allowed to ever infiltrate others' secrets by using any method,
    even the government. We don't go naked out the street because of privacy,
    We are not allowed to get inside others' home, even polices cannot, 
    for the same reason. And this human right should be always regarded, no matter it
    is a physical or electonical media, unless it is legal to steal things
    on internet but robbing others' house remain a crime. \par
    The second reason is handling privacy to total control of government is 
    highly dengerous, both in a political and techincal sense. Techincal, a government
    that stores tons of keys of citizen maybe a highly valuable target to hackers
    , considering that most of the best hackers won't choose to work in government,
    these keys will proably got stolen by those highly intelligent guys.
    We can't put the stake of our privacy, which could mean business, into 
    an unsafe situation. In a political sense, it is highly possible that governments
    which holds the true power, use their ability without legal authorization,
    or even use it for the leaders' own political good. How can we trust them?
    Snowden has already expose that this is actually happening.
\end{answer}

\begin{question}{Q. 2}
	Use Fermat’s little theorem to find the remainder when $2^{123}$ is divided by $29$. 
\end{question}
\begin{answer}{Q. 2}
	\hfill \break
    Since $29$ is a prime number, we have
    $$ 2^{28} \equiv 1  \mod {29}$$
    hence we have
    \begin{align}
        2^{123} &\equiv 2^{28\times 4 + 11} \mod 29 \\
        2^{123} &\equiv (2^{28})^4  (2^{11}) \mod 29 \\
        2^{123} &\equiv  2^{11} \mod 29 \\
        2^{123} &\equiv  2048 \mod 29 \\
        2^{123} &\equiv  18 \mod 29 
    \end{align}
\end{answer}
% --------------------------------------------------------------
%     You don't have to mess with anything below this line.
% --------------------------------------------------------------

\end{document}