% --------------------------------------------------------------
% This is all preamble stuff that you don't have to worry about.
% Head down to where it says "Start here"
% --------------------------------------------------------------
 
\documentclass[12pt]{article}
 
    \usepackage[margin=1in]{geometry} 
    \usepackage{amsmath,amsthm,amssymb}
     
    \newcommand{\N}{\mathbb{N}}
    \newcommand{\Z}{\mathbb{Z}}
     
    \newenvironment{theorem}[2][Theorem]{\begin{trivlist}
    \item[\hskip \labelsep {\bfseries #1}\hskip \labelsep {\bfseries #2.}]}{\end{trivlist}}
    \newenvironment{lemma}[2][Lemma]{\begin{trivlist}
    \item[\hskip \labelsep {\bfseries #1}\hskip \labelsep {\bfseries #2.}]}{\end{trivlist}}
    \newenvironment{exercise}[2][Exercise]{\begin{trivlist}
    \item[\hskip \labelsep {\bfseries #1}\hskip \labelsep {\bfseries #2.}]}{\end{trivlist}}
    \newenvironment{reflection}[2][Reflection]{\begin{trivlist}
    \item[\hskip \labelsep {\bfseries #1}\hskip \labelsep {\bfseries #2.}]}{\end{trivlist}}
    \newenvironment{proposition}[2][Proposition]{\begin{trivlist}
    \item[\hskip \labelsep {\bfseries #1}\hskip \labelsep {\bfseries #2.}]}{\end{trivlist}}
    \newenvironment{corollary}[2][Corollary]{\begin{trivlist}
    \item[\hskip \labelsep {\bfseries #1}\hskip \labelsep {\bfseries #2.}]}{\end{trivlist}}
    \newenvironment{question}[2][Question]{\begin{trivlist}
    \item[\hskip \labelsep {\bfseries #1}\hskip \labelsep {\bfseries #2.}]}{\end{trivlist}}
    \newenvironment{answer}[2][Answer]{\begin{trivlist}
    \item[\hskip \labelsep {\bfseries #1}\hskip \labelsep {\bfseries #2.}]}{\end{trivlist}}
    
     
\begin{document}

% --------------------------------------------------------------
%                         Start here
% --------------------------------------------------------------

%\renewcommand{\qedsymbol}{\filledbox}

\title{Assignment 8}%replace X with the appropriate number
\author{Michael Lee\\ %replace with your name
	CCST9017 - Hidden Order in Daily Life: A Mathematical Perspective \\
	University Number 3035569110 \\
	Tutorial Group 009
} %if necessary, replace with your course title


\maketitle

\begin{question}{Q. 1}
	Please find the total number of errors (typos) in the book.
\end{question}
\begin{answer}{Q. 1}
	\hfill \break
	Let $N$ be the total number of typos int the book. \\
	Let those errors considered marked to be II's. \\
	Then by considering marked items, we have:
	\begin{align}
		15/30 & = 25/N \\
		N     & = 50
	\end{align}
	Hence total number of typos is 50.
\end{answer}

\begin{question}{Q. 2}
	Heuristic Reason for Benford’s Law
\end{question}
\begin{answer}{Q. 2}
	\hfill \break
	Considering that
	\begin{equation}
		100d(1+r\%)^{f(d)}=100(d+1)
	\end{equation}
	For $d$ is a interger between 1 to 9. \\
	We have
	\begin{align}
		d(1+r\%)^{f(d)} & =(d+1)                       \\
		(1+r\%)^{f(d)}  & =1+1/d                       \\
		f(d)            & = \log (1+1/d) / \log(1+r\%)
	\end{align}
	The total time $F$ for all digit changes is given by
	\begin{align}
		F & = \sum_{n=d}^9 f(d)                                    \\
		  & = \left(\sum_{n=d}^9 \log (1+1/d)\right) / \log(1+r\%) \\
		  & = \log 10 / \log (1+r\%)                               \\
		  & = 1 / \log (1+r\%)
	\end{align}
	Hence
	\begin{equation}
		P(\textrm{first digit} = d) = f(d)/F = \log (1+1/d)
	\end{equation}
\end{answer}
% --------------------------------------------------------------
%     You don't have to mess with anything below this line.
% --------------------------------------------------------------

\end{document}