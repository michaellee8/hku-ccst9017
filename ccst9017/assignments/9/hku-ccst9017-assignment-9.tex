% --------------------------------------------------------------
% This is all preamble stuff that you don't have to worry about.
% Head down to where it says "Start here"
% --------------------------------------------------------------
 
\documentclass[12pt]{article}
 
    \usepackage[margin=1in]{geometry} 
    \usepackage{amsmath,amsthm,amssymb}
     
    \newcommand{\N}{\mathbb{N}}
    \newcommand{\Z}{\mathbb{Z}}
     
    \newenvironment{theorem}[2][Theorem]{\begin{trivlist}
    \item[\hskip \labelsep {\bfseries #1}\hskip \labelsep {\bfseries #2.}]}{\end{trivlist}}
    \newenvironment{lemma}[2][Lemma]{\begin{trivlist}
    \item[\hskip \labelsep {\bfseries #1}\hskip \labelsep {\bfseries #2.}]}{\end{trivlist}}
    \newenvironment{exercise}[2][Exercise]{\begin{trivlist}
    \item[\hskip \labelsep {\bfseries #1}\hskip \labelsep {\bfseries #2.}]}{\end{trivlist}}
    \newenvironment{reflection}[2][Reflection]{\begin{trivlist}
    \item[\hskip \labelsep {\bfseries #1}\hskip \labelsep {\bfseries #2.}]}{\end{trivlist}}
    \newenvironment{proposition}[2][Proposition]{\begin{trivlist}
    \item[\hskip \labelsep {\bfseries #1}\hskip \labelsep {\bfseries #2.}]}{\end{trivlist}}
    \newenvironment{corollary}[2][Corollary]{\begin{trivlist}
    \item[\hskip \labelsep {\bfseries #1}\hskip \labelsep {\bfseries #2.}]}{\end{trivlist}}
    \newenvironment{question}[2][Question]{\begin{trivlist}
    \item[\hskip \labelsep {\bfseries #1}\hskip \labelsep {\bfseries #2.}]}{\end{trivlist}}
    \newenvironment{answer}[2][Answer]{\begin{trivlist}
    \item[\hskip \labelsep {\bfseries #1}\hskip \labelsep {\bfseries #2.}]}{\end{trivlist}}
    
     
\begin{document}

% --------------------------------------------------------------
%                         Start here
% --------------------------------------------------------------

%\renewcommand{\qedsymbol}{\filledbox}

\title{Assignment 9}%replace X with the appropriate number
\author{Michael Lee\\ %replace with your name
	CCST9017 - Hidden Order in Daily Life: A Mathematical Perspective \\
	University Number 3035569110 \\
	Tutorial Group 009
} %if necessary, replace with your course title


\maketitle

\begin{question}{Q. 1}
	Show that the 13-digit ISBN13 system (a) can detect all 1-errors, (b)
	but not all 2-errors come from transposition of digits.
\end{question}
\begin{answer}{Q. 1}
	\hfill \break
	Let such ISBN13 code be $n=n_1 n_2 n_3 \hdots n_{13}$. \\
	Let the error digit be $n_i$, which is mistakenly changed to $m_i$, where $n_i\neq m_i$\\
	Let the new (error) value be $T$, and original (correct) value be $S$\\ \\
	Consider case 1, $i$ is a odd number, \\
	Then we have $T-S=1\times m_i -1 \times n_i = m_i-n_i$ \\
	Since $0\leq m_i,n_i\leq 9$ and $m_i\neq n_i$, we must have $m_i-n_i \not\equiv 0 \pmod {10}$ \\ \\
	Consider case 2, $i$ is a even number, \\
	Then we have $T-S=3\times m_i - 3 \times n_i = 3\left(m_i-n_i\right)$ \\
	Since $m_i-n_i \not\equiv 0 \pmod {10}$ and $3 \perp 10$, we must have $3\left(m_i-n_i\right) \not\equiv 0 \pmod {10}$ \\ \\
	Because all ISBN13 code with 1-digit error will not satsify the detection condition,\\
	Therefore ISBN13 can detect all 1-digit errors. \\ \\
	Consider a transposition of 2-digits, $n_i$ and $n_{i+1}$, \\
	Then we have $T-S=\pm \left( 3 n_i+n_{i+1}-n_i-3n_{i+1}\right)= \pm 2 \left( n_i-n_{i+1} \right)$ \\
	Therfore, we have 1 example for it: 2770000000000 have the first and second digit swapped into 7270000000000. \\
	The sum of first code is $2+7\times 3 + 7=30$, which is divisible by 10. \\
	The sum of second code is $7+2\times 3+7=20$, which is also devisible by 10. \\
	Therefore, it cannot check all 2-digit errors due to transposition.
\end{answer}

\begin{question}{Q. 2}
	Show that the codeword a cannot be a valid codeword.
\end{question}
\begin{answer}{Q. 2}
	\hfill \break
	According to the lecture notes, the minimum	distance of Hamming (7,4) code is 3. \\
	Therefore, for any valid code $d$, there does not exists a valid code that have a distance less than 3. \\
	Hence $a$, which only differs from $d$ by 2 bits, cannot be a valid one.
\end{answer}
% --------------------------------------------------------------
%     You don't have to mess with anything below this line.
% --------------------------------------------------------------

\end{document}