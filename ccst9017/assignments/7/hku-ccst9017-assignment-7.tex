% --------------------------------------------------------------
% This is all preamble stuff that you don't have to worry about.
% Head down to where it says "Start here"
% --------------------------------------------------------------
 
\documentclass[12pt]{article}
 
    \usepackage[margin=1in]{geometry} 
    \usepackage{amsmath,amsthm,amssymb}
     
    \newcommand{\N}{\mathbb{N}}
    \newcommand{\Z}{\mathbb{Z}}
     
    \newenvironment{theorem}[2][Theorem]{\begin{trivlist}
    \item[\hskip \labelsep {\bfseries #1}\hskip \labelsep {\bfseries #2.}]}{\end{trivlist}}
    \newenvironment{lemma}[2][Lemma]{\begin{trivlist}
    \item[\hskip \labelsep {\bfseries #1}\hskip \labelsep {\bfseries #2.}]}{\end{trivlist}}
    \newenvironment{exercise}[2][Exercise]{\begin{trivlist}
    \item[\hskip \labelsep {\bfseries #1}\hskip \labelsep {\bfseries #2.}]}{\end{trivlist}}
    \newenvironment{reflection}[2][Reflection]{\begin{trivlist}
    \item[\hskip \labelsep {\bfseries #1}\hskip \labelsep {\bfseries #2.}]}{\end{trivlist}}
    \newenvironment{proposition}[2][Proposition]{\begin{trivlist}
    \item[\hskip \labelsep {\bfseries #1}\hskip \labelsep {\bfseries #2.}]}{\end{trivlist}}
    \newenvironment{corollary}[2][Corollary]{\begin{trivlist}
    \item[\hskip \labelsep {\bfseries #1}\hskip \labelsep {\bfseries #2.}]}{\end{trivlist}}
    \newenvironment{question}[2][Question]{\begin{trivlist}
    \item[\hskip \labelsep {\bfseries #1}\hskip \labelsep {\bfseries #2.}]}{\end{trivlist}}
    \newenvironment{answer}[2][Answer]{\begin{trivlist}
    \item[\hskip \labelsep {\bfseries #1}\hskip \labelsep {\bfseries #2.}]}{\end{trivlist}}
    
     
\begin{document}

% --------------------------------------------------------------
%                         Start here
% --------------------------------------------------------------

%\renewcommand{\qedsymbol}{\filledbox}

\title{Assignment 7}%replace X with the appropriate number
\author{Michael Lee\\ %replace with your name
	CCST9017 - Hidden Order in Daily Life: A Mathematical Perspective \\
	University Number 3035569110 \\
	Tutorial Group 009
} %if necessary, replace with your course title


\maketitle

\begin{question}{Q. 1}
	"Flush" in card game
\end{question}
\begin{answer}{Q. 1}\end{answer}
Consider if you discard the 2 clubs, then only the 3 hearts remain. Therefore, the only possible flush is a hearts flush consists of 5 hearts. \\
Hence the possibility of getting a flush is equal to that of getting 2 hearts from the remaining 47 cards, which is given by:
\begin{equation}
	10/47 \times 9/46 = 0.0416
\end{equation}

\begin{question}{Q. 2}
	Birthday problem
\end{question}
\begin{answer}{Q. 2}\end{answer}
Let such probability be $P$, number of required people be $n$.  \\
Consider the birthday of such person $i$ be $D_i$, where $1\leq D_i \leq 365$  \\
Let's specify the "specific person" to be the first one, hence having birthday $D_1$  \\
For $D_2, D_3 , \dots , D_n \neq D_1$ not equal to $D_1$, there are $365-1$ choices out of $365$ days  \\
The possibility of $D_2, D_3 , \dots , D_n \neq D_1$ should hence be given by:
\begin{equation}
    \left(\frac{364}{365}\right)^{n - 1}
\end{equation}
Hence the possibility of any one of $D_2, D_3 , \dots , D_n$ is equal to $D_1$, which is $P$, is given by:  
\begin{equation}
    1-\left(\frac{364}{365}\right)^{n - 1}
\end{equation}
By solving $P>0.5$, we have
\begin{equation}
    n > 253.652
\end{equation}
Hence the minimum $n$ is $254$

% --------------------------------------------------------------
%     You don't have to mess with anything below this line.
% --------------------------------------------------------------

\end{document}