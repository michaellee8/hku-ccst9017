% --------------------------------------------------------------
% This is all preamble stuff that you don't have to worry about.
% Head down to where it says "Start here"
% --------------------------------------------------------------
 

\documentclass[12pt]{article}
    \usepackage[margin=1in]{geometry} 
	\usepackage{amsmath,amsthm,amssymb,bm}
	\makeatletter
\renewcommand*\env@matrix[1][*\c@MaxMatrixCols c]{%
  \hskip -\arraycolsep
  \let\@ifnextchar\new@ifnextchar
  \array{#1}}
\makeatother
     
    \newcommand{\N}{\mathbb{N}}
    \newcommand{\Z}{\mathbb{Z}}
    \newcommand{\matr}[1]{\bm{#1}}
    \newenvironment{theorem}[2][Theorem]{\begin{trivlist}
    \item[\hskip \labelsep {\bfseries #1}\hskip \labelsep {\bfseries #2.}]}{\end{trivlist}}
    \newenvironment{lemma}[2][Lemma]{\begin{trivlist}
    \item[\hskip \labelsep {\bfseries #1}\hskip \labelsep {\bfseries #2.}]}{\end{trivlist}}
    \newenvironment{exercise}[2][Exercise]{\begin{trivlist}
    \item[\hskip \labelsep {\bfseries #1}\hskip \labelsep {\bfseries #2.}]}{\end{trivlist}}
    \newenvironment{reflection}[2][Reflection]{\begin{trivlist}
    \item[\hskip \labelsep {\bfseries #1}\hskip \labelsep {\bfseries #2.}]}{\end{trivlist}}
    \newenvironment{proposition}[2][Proposition]{\begin{trivlist}
    \item[\hskip \labelsep {\bfseries #1}\hskip \labelsep {\bfseries #2.}]}{\end{trivlist}}
    \newenvironment{corollary}[2][Corollary]{\begin{trivlist}
    \item[\hskip \labelsep {\bfseries #1}\hskip \labelsep {\bfseries #2.}]}{\end{trivlist}}
    \newenvironment{question}[2][Question]{\begin{trivlist}
    \item[\hskip \labelsep {\bfseries #1}\hskip \labelsep {\bfseries #2.}]}{\end{trivlist}}
    \newenvironment{answer}[2][Answer]{\begin{trivlist}
    \item[\hskip \labelsep {\bfseries #1}\hskip \labelsep {\bfseries #2.}]}{\end{trivlist}}
     
    \begin{document}
    
% --------------------------------------------------------------
%                         Start here
% --------------------------------------------------------------
    
%\renewcommand{\qedsymbol}{\filledbox}
    
\title{Assignment 2}%replace X with the appropriate number
\author{Michael Lee\\ %replace with your name
	MATH1853 \\
	University Number 3035569110 
} %if necessary, replace with your course title
    
    
\maketitle
\begin{answer}{Q. 1}
\end{answer}
For eigenvalues of $A$, we have
\begin{equation}
	|\matr A-\lambda \matr I| = 0
\end{equation}
with
\begin{align}
	          &            
	\begin{bmatrix}
	a         & b          \\
	b         & -a         \\ 
	\end{bmatrix}
	-
	\begin{bmatrix}
	\lambda   & 0          \\
	0         & \lambda    \\ 
	\end{bmatrix} \\
	=         &            
	\begin{bmatrix}
	a-\lambda & b          \\
	b         & -a-\lambda \\
	\end{bmatrix}
\end{align}
hence 
\begin{align}
	\begin{vmatrix}
	a-\lambda                                                     & b                    \\
	b                                                             & -a-\lambda           \\
	\end{vmatrix}
	                                                              & =0                   \\\   
	-\left(\alpha + \lambda\right)\left(\alpha-\lambda\right)-b^2 & = 0                  
	-a^2+\lambda^2-b^2                                            & = 0                  \\
	\lambda^2                                                     & = a^2+b^2            \\
	\lambda                                                       & = \pm \sqrt{a^2+b^2} 
\end{align}
For eigenvalue $\sqrt{a^2+b^2}$ and eigenvector $v_1$ of $\matr A$, we have
\begin{align}
	\left(\matr A-\sqrt{a^2+b^2} \matr I\right) \matr v_1 & = 0 \\
	\begin{bmatrix}
	a + \sqrt{a^{2} + b^{2}} & b                          &   \\
	b                        & - a + \sqrt{a^{2} + b^{2}} &   
	\end{bmatrix} 
	\matr v_1                             & =0  \\
\end{align}
Hence we have 
\begin{align}
	\begin{bmatrix}[cc|c]
	a + \sqrt{a^{2} + b^{2}} & b                                                                        & 0 \\
	b                        & - a + \sqrt{a^{2} + b^{2}}                                               & 0 \\
	\end{bmatrix} \\
	\begin{bmatrix}[cc|c]
	1                        & \frac b {a + \sqrt{a^{2} + b^{2}}}                                       & 0 \\
	1                        & \frac {- a + \sqrt{a^{2} + b^{2}}}b                                      & 0 \\
	\end{bmatrix} \\
	\begin{bmatrix}[cc|c]
	0                        & \frac b {a + \sqrt{a^{2} + b^{2}}} - \frac {- a + \sqrt{a^{2} + b^{2}}}b & 0 \\
	2                        & \frac b {a + \sqrt{a^{2} + b^{2}}} + \frac {- a + \sqrt{a^{2} + b^{2}}}b & 0 \\
	\end{bmatrix} \\
	\begin{bmatrix}[cc|c]
	0                      & 
	\frac {b\left(\sqrt{a^{2} + b^{2}}-a \right)} 
	{\left(a + \sqrt{a^{2} + b^{2}}\right)\left(\sqrt{a^{2} + b^{2}}-a \right)} 
	- \frac {- a + \sqrt{a^{2} + b^{2}}}b 
	& 0 \\
	2                        & 
	\frac {b\left(\sqrt{a^{2} + b^{2}}-a \right)} 
	{\left(a + \sqrt{a^{2} + b^{2}}\right)\left(\sqrt{a^{2} + b^{2}}-a \right)} 
	+ \frac {- a + \sqrt{a^{2} + b^{2}}}b 
	& 0 \\                                                                       
	\end{bmatrix} \\
	\begin{bmatrix}[cc|c]
	0                      & 
	\frac {\sqrt{a^{2} + b^{2}}-a} 
	{b} 
	- \frac {- a + \sqrt{a^{2} + b^{2}}}b 
	& 0 \\
	2                        & 
	\frac {\sqrt{a^{2} + b^{2}}-a} 
	{b}
	+ \frac {- a + \sqrt{a^{2} + b^{2}}}b 
	& 0 \\                                                                       
	\end{bmatrix} \\
	\begin{bmatrix}[cc|c]
	1                        & \frac  {a + \sqrt{a^{2} + b^{2}}}    b                                   & 0 \\
	0                        & 0                                                                        & 0 \\
	\end{bmatrix} 
\end{align}
Therefore
\begin{equation}
	v_1=\begin{bmatrix}
	1 \\
	- \frac  {a + \sqrt{a^{2} + b^{2}}} b
	\end{bmatrix}
\end{equation} 

For eigenvalue $-\sqrt{a^2+b^2}$ and eigenvector $v_2$ of $\matr A$, we have
\begin{align}
	\left(\matr A+\sqrt{a^2+b^2} \matr I\right) \matr v_1 & = 0 \\
	\begin{bmatrix}
	a - \sqrt{a^{2} + b^{2}} & b                          &   \\
	b                        & - a - \sqrt{a^{2} + b^{2}} &   
	\end{bmatrix} 
	\matr v_1                             & =0  
\end{align}
Similar from above, we have
\begin{align}
	\begin{bmatrix}[cc|c]
	a - \sqrt{a^{2} + b^{2}} & b                                       & 0 \\
	b                        & - a - \sqrt{a^{2} + b^{2}}              & 0 \\
	\end{bmatrix} \\
	\begin{bmatrix}[cc|c]
	1                        & \frac  {-a - \sqrt{a^{2} + b^{2}}}    b & 0 \\
	0                        & 0                                       & 0 \\
	\end{bmatrix} \\
\end{align}
Therefore
\begin{equation}
	v_2=\begin{bmatrix}
	1 \\
	\frac  {a + \sqrt{a^{2} + b^{2}}} b
	\end{bmatrix} \\
\end{equation} 
Eigenpairs of $\matr A$ are $\left(
\sqrt{a^2+b^2},
\begin{bmatrix}
	1                                     \\
	- \frac  {a + \sqrt{a^{2} + b^{2}}} b 
\end{bmatrix}
\right)$ and $\left(
-\sqrt{a^2+b^2},
\begin{bmatrix}
	1                                   \\ 
	\frac  {a + \sqrt{a^{2} + b^{2}}} b 
\end{bmatrix}
\right)$\\
Since all the eigenvalues are distinct, $\matr A$ is diagonizable. \\ \\


For eigenvalues of B, we have
\begin{equation}
	|\matr B - \lambda \matr I| = 0
\end{equation}
with
\begin{align}
	            &             
	\begin{bmatrix}
	a           & -b          \\
	b           & a           \\ 
	\end{bmatrix}
	-
	\begin{bmatrix}
	\lambda     & 0           \\
	0           & \lambda     \\ 
	\end{bmatrix} \\
	=           &             
	\begin{bmatrix}
	a - \lambda & - b         \\
	b           & a - \lambda \\
	\end{bmatrix}
\end{align}
hence
\begin{align}
	b^2 + \left(a-\lambda\right)^2 & = 0         \\
	\left(a-\lambda\right)^2       & = -b^2      \\
	a-\lambda                      & = \pm b i   \\
	\lambda                        & = a \pm b i 
\end{align}


For eigenvalue $a+bi$ and eigenvector $u_1$ for $\matr B$, we have
\begin{align}
	\left[\matr B - \left(a+bi\right) \matr I \right] u_1 &= 0 \\
	b \left[\begin{matrix}i & -1 \\1 & i\end{matrix}\right] u_1&=0 \\
\end{align}
\begin{align}
	\begin{bmatrix}[cc|c]
	i          & -1           & 0 \\
	1          & i            & 0 \\
	\end{bmatrix} \\
	\begin{bmatrix}[cc|c]
	i+1        & -1+i         & 0 \\
	1          & i            & 0 \\
	\end{bmatrix} \\
	\begin{bmatrix}[cc|c]
	(i+1)(i-1) & (-1+i)(-1+i) & 0 \\
	1          & i            & 0 \\
	\end{bmatrix} \\
	\begin{bmatrix}[cc|c]\sqrt{a^2+b^2}
	-1-1       & -2i          & 0 \\
	1          & i            & 0 \\
	\end{bmatrix} \\
	\begin{bmatrix}[cc|c]
	2          & 2i           & 0 \\
	1          & i            & 0 \\
	\end{bmatrix} \\
	\begin{bmatrix}[cc|c]
	1          & i            & 0 \\
	0          & 0            & 0 \\
	\end{bmatrix} \\
	\therefore u_1 = \begin{bmatrix}1 \\ -i \end{bmatrix}
\end{align}

For eigenvalue $a-bi$ and eigenvector $u_2$ for $\matr B$, we have
\begin{align}
	\left[\matr B - \left(a-bi\right) \matr I \right] u_2 &= 0 \\
	b \left[\begin{matrix}-i & -1 \\1 & -i\end{matrix}\right] u_2&=0 \\
\end{align}
Using similar method from above, we will have
\begin{equation}
	u_2 = \begin{bmatrix}-1 \\ i \end{bmatrix}
\end{equation}
Eigenpairs of $\matr B$ are $\left(
a+bi,
\begin{bmatrix}1 \\ -i \end{bmatrix}
\right)$ and $\left(
a-bi,
\begin{bmatrix}-1 \\ i \end{bmatrix}
\right)$\\ 
Since all the eigenvalues are distinct, $\matr B$ is diagonizable. \\ \\

\begin{answer}{Q. 2}\end{answer}
Let $\matr v$ be the vector provided, consider that
\begin{align}
    \matr v =& \begin{bmatrix}
    4 \\ 2 \\3
    \end{bmatrix} \\
    =& -2 \begin{bmatrix}
    1 \\ -1 \\ 0
    \end{bmatrix}
    +3 \begin{bmatrix}
    2 \\ 0 \\ 1
    \end{bmatrix} \\
    =& -2 \matr v_1 + 3 \matr v_2
\end{align}
Hence we have
\begin{align}
   & \matr A ^{999} \begin{bmatrix}
    4 \\ 2 \\3
    \end{bmatrix} \\
    =& \matr A^ {999} \matr v \\
    =&-2\matr A ^{999} \matr v_1 + 3 \matr A^{999} \matr v_2 \\
    =& -2 \lambda _1 ^{999}\matr v _1 + 3 \lambda _2^{999} \matr v_2 \\
    =& -2 1 ^{999}\matr v _1 + 3 (-1)^{999} \matr v_2 \\
    =& -2 \matr v _1 -3 \matr v _2 \\
    =& -2 \begin{bmatrix}
    1 \\ -1 \\ 0
    \end{bmatrix} -3 \begin{bmatrix}
    2 \\ 0 \\1
    \end{bmatrix} \\
    =&\begin{bmatrix}
    -8 \\ 2 \\ -3
    \end{bmatrix}
\end{align}



\begin{answer}{Q. 3}\end{answer}
Considering that the eigenvector $\matr v_i$ of matrix $\matr A$ with eigenvalue $\matr v_i$ is given by
\begin{equation}
	\left(\matr A - \lambda_i \matr I \right)\matr v_i = 0 
\end{equation}
and a vector $\matr v$ is considered a null space vector of $\matr A$ only if
\begin{equation}
	\matr A \matr v = 0
\end{equation}
Hence if $\lambda_i = 0$, then we must have
\begin{align}
	(\matr A - 0 \matr I) \matr v _i & = 0 \\
	\matr A \matr v _ i              & = 0 
\end{align}
$\therefore$ $\matr A$ has a nullspace vector $v_i$ if it has a eigenvalue of $\lambda _i=0$\\ \\

If there is no such $\lambda_i=0$, then for all $\matr v_i$
\begin{equation}
	\matr A \matr v_i = \lambda_i \matr v _i \neq 0
\end{equation}
$\therefore$ By (86), no null space vector exists. \\
$\therefore$ $\matr A$ has a null space vector $v_i$ only if it has a eigenvalue of $\lambda _i=0$ \\\\
$\because$ $p(\lambda)$ is of degree 6 \\
$\therefore$ $\matr A$ has a size of $6\times 6$ \\ \\
By rank\textendash nullity theorem,
\begin{equation}
    \textrm{Rank}(\matr A) + \textrm{Nullity}(\matr A) = \textrm{dim} \matr A
\end{equation}
$\because$ No solution of $p(\lambda)=0$ is 0 \\
$\therefore$ By the above proved statement, $A$ have no null space vector \\
$\therefore$ $\textrm{Nullity}(\matr A)=0$ \\
$\therefore$ By (90), rank of $\matr A$ is given by $\textrm{Rank}(\matr A) = \textrm{dim} \matr A = 0$ \\ \\

\begin{answer}{Q. 4}\end{answer}
Let $\matr M$ be the matrix, hence
\begin{align}
    |\matr M - \lambda \matr I| =&  
    \begin{bmatrix}
    -\lambda & 0 & 0 & 0 \\
    1 & 2- \lambda & 3 & 4 \\
    0 & 0 & -\lambda & 0 \\
    1 & 2 & 3 & 4-\lambda
    \end{bmatrix} \\
    =&\lambda ^2((2-\lambda)(4-\lambda)-8) \\ 
    =& \lambda ^3(\lambda -6)
\end{align}
Hence by solving $\lambda ^3 (\lambda -6)=0$, \\
we have eigenvalue $0$ of algebraic multiplicity 3 \\
and eigenvalue $6$ of algebraic multiplicity 1. \\ \\
For $\lambda = 0$,
\begin{align}
    \begin{bmatrix}[cccc|c]
    0 & 0 & 0 & 0& 0 \\
    1 & 2 & 3 & 4& 0 \\
    0 & 0 & 0 & 0& 0 \\
    1 & 2 & 3 & 4& 0
    \end{bmatrix} \\
    \begin{bmatrix}[cccc|c]
    1 & 2 & 3 & 4& 0 \\
    0 & 0 & 0 & 0& 0 \\
    0 & 0 & 0 & 0& 0 \\
    0 & 0 & 0 & 0& 0 \\
    \end{bmatrix} 
\end{align}
Hence eigenvectors of 0 are
\begin{equation}
    \left [ \left[\begin{matrix}-2\\1\\0\\0\end{matrix}\right], \quad \left[\begin{matrix}-3\\0\\1\\0\end{matrix}\right], \quad \left[\begin{matrix}-4\\0\\0\\1\end{matrix}\right]\right ]
\end{equation} 
Which mean it has linear multiplicity of 3\\ \\
For $\lambda=6$, \\
For a eigenvalue of algebraic multiplicity 1, its linear multiplicity must be larger than 0, since at least 1 eigenvectors exists, and its eigenvalue must be equal or smaller than 1, since linear multiplicity must be smaller or equal to algebraic multiplicity. Hence its linear multiplicity is 1 as well.

\begin{answer}{Q. 5}\end{answer}
Let $\matr v = \begin{bmatrix}
a_1 \\ a_2 \\ \vdots \\ a_n
\end{bmatrix}$
Then we have
\begin{align}
    \matr v \matr v ^T =&\begin{bmatrix}
a_1 \\ a_2 \\ \vdots \\ a_n
\end{bmatrix} \begin{bmatrix}
a_1 & a_2 & \hdots & a_n
\end{bmatrix} \\
=&\begin{bmatrix}
a_1^2 & a_1  a_2 & \hdots & a_1 a^ n \\
a_2 a_1 &   a_2^2 & \hdots & a_2 a^ n \\
\vdots & \vdots & \ddots & \vdots \\
a_n a_1 & a_n a_2 & \hdots & a_n^2
\end{bmatrix}
\end{align}
Therefore $\matr v \matr v ^T$, hence $-\frac 2 {||v||^2} \matr v \matr v ^T$ are symmetric. \\
Since $\matr I$ is symmetric as well, \\
$\matr M = \matr I -\frac 2 {||v||^2} \matr v \matr v ^T $ is also symmetric. \\ \\
\begin{align}
    \matr M^2 =& (\matr I -\frac 2 {||\matr v||^2} \matr v \matr v ^T)^2 \\
    =& \matr I - 4 \frac {\matr v \matr v ^T}{||v||^2} + 4 \frac 1 {||\matr v||^4} \matr v (||\matr v||^2) \matr v ^T \\
    =& \matr I \\
    \matr M =& \matr M ^{-1}
\end{align}
% --------------------------------------------------------------
%     You don't have to mess with anything below this line.
% --------------------------------------------------------------
\end{document}