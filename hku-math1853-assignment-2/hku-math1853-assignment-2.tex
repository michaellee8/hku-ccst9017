% --------------------------------------------------------------
% This is all preamble stuff that you don't have to worry about.
% Head down to where it says "Start here"
% --------------------------------------------------------------
 

\documentclass[12pt]{article}
    \usepackage[margin=1in]{geometry} 
	\usepackage{amsmath,amsthm,amssymb}
	\makeatletter
\renewcommand*\env@matrix[1][*\c@MaxMatrixCols c]{%
  \hskip -\arraycolsep
  \let\@ifnextchar\new@ifnextchar
  \array{#1}}
\makeatother
     
    \newcommand{\N}{\mathbb{N}}
    \newcommand{\Z}{\mathbb{Z}}
    \newcommand{\matr}[1]{\mathbf{#1}}
    \newenvironment{theorem}[2][Theorem]{\begin{trivlist}
    \item[\hskip \labelsep {\bfseries #1}\hskip \labelsep {\bfseries #2.}]}{\end{trivlist}}
    \newenvironment{lemma}[2][Lemma]{\begin{trivlist}
    \item[\hskip \labelsep {\bfseries #1}\hskip \labelsep {\bfseries #2.}]}{\end{trivlist}}
    \newenvironment{exercise}[2][Exercise]{\begin{trivlist}
    \item[\hskip \labelsep {\bfseries #1}\hskip \labelsep {\bfseries #2.}]}{\end{trivlist}}
    \newenvironment{reflection}[2][Reflection]{\begin{trivlist}
    \item[\hskip \labelsep {\bfseries #1}\hskip \labelsep {\bfseries #2.}]}{\end{trivlist}}
    \newenvironment{proposition}[2][Proposition]{\begin{trivlist}
    \item[\hskip \labelsep {\bfseries #1}\hskip \labelsep {\bfseries #2.}]}{\end{trivlist}}
    \newenvironment{corollary}[2][Corollary]{\begin{trivlist}
    \item[\hskip \labelsep {\bfseries #1}\hskip \labelsep {\bfseries #2.}]}{\end{trivlist}}
    \newenvironment{question}[2][Question]{\begin{trivlist}
    \item[\hskip \labelsep {\bfseries #1}\hskip \labelsep {\bfseries #2.}]}{\end{trivlist}}
    \newenvironment{answer}[2][Answer]{\begin{trivlist}
    \item[\hskip \labelsep {\bfseries #1}\hskip \labelsep {\bfseries #2.}]}{\end{trivlist}}
     
    \begin{document}
    
% --------------------------------------------------------------
%                         Start here
% --------------------------------------------------------------
    
%\renewcommand{\qedsymbol}{\filledbox}
    
\title{Assignment 2}%replace X with the appropriate number
\author{Michael Lee\\ %replace with your name
	MATH1853 \\
	University Number 3035569110 
} %if necessary, replace with your course title
    
    
\maketitle
\begin{answer}{Q. 1}
\end{answer}
For eigenvalues of $A$, we have
\begin{equation}
	|\matr A-\lambda \matr I| = 0
\end{equation}
with
\begin{align}
	          &            
	\begin{bmatrix}
	a         & b          \\
	b         & -a         \\ 
	\end{bmatrix}
	-
	\begin{bmatrix}
	\lambda   & 0          \\
	0         & \lambda    \\ 
	\end{bmatrix} \\
	=         &            
	\begin{bmatrix}
	a-\lambda & b          \\
	b         & -a-\lambda \\
	\end{bmatrix}
\end{align}
hence 
\begin{align}
	\begin{vmatrix}
	a-\lambda                                                     & b                    \\
	b                                                             & -a-\lambda           \\
	\end{vmatrix}
	                                                              & =0                   \\\   
	-\left(\alpha + \lambda\right)\left(\alpha-\lambda\right)-b^2 & = 0                  
	-a^2+\lambda^2-b^2                                            & = 0                  \\
	\lambda^2                                                     & = a^2+b^2            \\
	\lambda                                                       & = \pm \sqrt{a^2+b^2} 
\end{align}
For eigenvalue $\sqrt{a^2+b^2}$ and eigenvector $v_1$ of $\matr A$, we have
\begin{align}
	\left(\matr A-\sqrt{a^2+b^2} \matr I\right) \matr v_1 & = 0 \\
	\begin{bmatrix}
	a + \sqrt{a^{2} + b^{2}} & b                          &   \\
	b                        & - a + \sqrt{a^{2} + b^{2}} &   
	\end{bmatrix} 
	\matr v_1                             & =0  \\
\end{align}
Hence we have 
\begin{align}
	\begin{bmatrix}[cc|c]
	a + \sqrt{a^{2} + b^{2}} & b                                                                        & 0 \\
	b                        & - a + \sqrt{a^{2} + b^{2}}                                               & 0 \\
	\end{bmatrix} \\
	\begin{bmatrix}[cc|c]
	1                        & \frac b {a + \sqrt{a^{2} + b^{2}}}                                       & 0 \\
	1                        & \frac {- a + \sqrt{a^{2} + b^{2}}}b                                      & 0 \\
	\end{bmatrix} \\
	\begin{bmatrix}[cc|c]
	0                        & \frac b {a + \sqrt{a^{2} + b^{2}}} - \frac {- a + \sqrt{a^{2} + b^{2}}}b & 0 \\
	2                        & \frac b {a + \sqrt{a^{2} + b^{2}}} + \frac {- a + \sqrt{a^{2} + b^{2}}}b & 0 \\
	\end{bmatrix} \\
	\begin{bmatrix}[cc|c]
	0                      & 
	\frac {b\left(\sqrt{a^{2} + b^{2}}-a \right)} 
	{\left(a + \sqrt{a^{2} + b^{2}}\right)\left(\sqrt{a^{2} + b^{2}}-a \right)} 
	- \frac {- a + \sqrt{a^{2} + b^{2}}}b 
	& 0 \\
	2                        & 
	\frac {b\left(\sqrt{a^{2} + b^{2}}-a \right)} 
	{\left(a + \sqrt{a^{2} + b^{2}}\right)\left(\sqrt{a^{2} + b^{2}}-a \right)} 
	+ \frac {- a + \sqrt{a^{2} + b^{2}}}b 
	& 0 \\                                                                       
	\end{bmatrix} \\
	\begin{bmatrix}[cc|c]
	0                      & 
	\frac {\sqrt{a^{2} + b^{2}}-a} 
	{b} 
	- \frac {- a + \sqrt{a^{2} + b^{2}}}b 
	& 0 \\
	2                        & 
	\frac {\sqrt{a^{2} + b^{2}}-a} 
	{b}
	+ \frac {- a + \sqrt{a^{2} + b^{2}}}b 
	& 0 \\                                                                       
	\end{bmatrix} \\
	\begin{bmatrix}[cc|c]
	1                        & \frac  {a + \sqrt{a^{2} + b^{2}}}    b                                   & 0 \\
	0                        & 0                                                                        & 0 \\
	\end{bmatrix} 
\end{align}
Therefore
\begin{equation}
	v_1=\begin{bmatrix}
	1 \\
	- \frac  {a + \sqrt{a^{2} + b^{2}}} b
	\end{bmatrix}
\end{equation} 

For eigenvalue $-\sqrt{a^2+b^2}$ and eigenvector $v_2$ of $\matr A$, we have
\begin{align}
	\left(\matr A+\sqrt{a^2+b^2} \matr I\right) \matr v_1 & = 0 \\
	\begin{bmatrix}
	a - \sqrt{a^{2} + b^{2}} & b                          &   \\
	b                        & - a - \sqrt{a^{2} + b^{2}} &   
	\end{bmatrix} 
	\matr v_1                             & =0  
\end{align}
Similar from above, we have
\begin{align}
	\begin{bmatrix}[cc|c]
	a - \sqrt{a^{2} + b^{2}} & b                                       & 0 \\
	b                        & - a - \sqrt{a^{2} + b^{2}}              & 0 \\
	\end{bmatrix} \\
	\begin{bmatrix}[cc|c]
	1                        & \frac  {-a - \sqrt{a^{2} + b^{2}}}    b & 0 \\
	0                        & 0                                       & 0 \\
	\end{bmatrix} \\
\end{align}
Therefore
\begin{equation}
	v_2=\begin{bmatrix}
	1 \\
	\frac  {a + \sqrt{a^{2} + b^{2}}} b
	\end{bmatrix} \\
\end{equation} 
Eigenpairs of $\matr A$ are $\left(
\sqrt{a^2+b^2},
\begin{bmatrix}
	1                                     \\
	- \frac  {a + \sqrt{a^{2} + b^{2}}} b 
\end{bmatrix}
\right)$ and $\left(
-\sqrt{a^2+b^2},
\begin{bmatrix}
	1                                   \\ 
	\frac  {a + \sqrt{a^{2} + b^{2}}} b 
\end{bmatrix}
\right)$\\
Since all the eigenvalues are distinct, $\matr A$ is diagonizable. \\ \\


For eigenvalues of B, we have
\begin{equation}
	|\matr B - \lambda \matr I| = 0
\end{equation}
with
\begin{align}
	            &             
	\begin{bmatrix}
	a           & -b          \\
	b           & a           \\ 
	\end{bmatrix}
	-
	\begin{bmatrix}
	\lambda     & 0           \\
	0           & \lambda     \\ 
	\end{bmatrix} \\
	=           &             
	\begin{bmatrix}
	a - \lambda & - b         \\
	b           & a - \lambda \\
	\end{bmatrix}
\end{align}
hence
\begin{align}
	b^2 + \left(a-\lambda\right)^2 & = 0         \\
	\left(a-\lambda\right)^2       & = -b^2      \\
	a-\lambda                      & = \pm b i   \\
	\lambda                        & = a \pm b i 
\end{align}


For eigenvalue $a+bi$ and eigenvector $u_1$ for $\matr B$, we have
\begin{align}
	\left[\matr B - \left(a+bi\right) \matr I \right] u_1 &= 0 \\
	b \left[\begin{matrix}i & -1 \\1 & i\end{matrix}\right] u_1&=0 \\
\end{align}
\begin{align}
	\begin{bmatrix}[cc|c]
	i          & -1           & 0 \\
	1          & i            & 0 \\
	\end{bmatrix} \\
	\begin{bmatrix}[cc|c]
	i+1        & -1+i         & 0 \\
	1          & i            & 0 \\
	\end{bmatrix} \\
	\begin{bmatrix}[cc|c]
	(i+1)(i-1) & (-1+i)(-1+i) & 0 \\
	1          & i            & 0 \\
	\end{bmatrix} \\
	\begin{bmatrix}[cc|c]\sqrt{a^2+b^2}
	-1-1       & -2i          & 0 \\
	1          & i            & 0 \\
	\end{bmatrix} \\
	\begin{bmatrix}[cc|c]
	2          & 2i           & 0 \\
	1          & i            & 0 \\
	\end{bmatrix} \\
	\begin{bmatrix}[cc|c]
	1          & i            & 0 \\
	0          & 0            & 0 \\
	\end{bmatrix} \\
	\therefore u_1 = \begin{bmatrix}1 \\ -i \end{bmatrix}
\end{align}

For eigenvalue $a-bi$ and eigenvector $u_2$ for $\matr B$, we have
\begin{align}
	\left[\matr B - \left(a-bi\right) \matr I \right] u_2 &= 0 \\
	b \left[\begin{matrix}-i & -1 \\1 & -i\end{matrix}\right] u_2&=0 \\
\end{align}
Using similar method from above, we will have
\begin{equation}
	u_2 = \begin{bmatrix}-1 \\ i \end{bmatrix}
\end{equation}
Eigenpairs of $\matr B$ are $\left(
a+bi,
\begin{bmatrix}1 \\ -i \end{bmatrix}
\right)$ and $\left(
a-bi,
\begin{bmatrix}-1 \\ i \end{bmatrix}
\right)$\\ 
Since all the eigenvalues are distinct, $\matr B$ is diagonizable. \\ \\

\begin{answer}{Q. 2}
\end{answer}
Let $\matr A$ be 
\begin{equation}
	\left[\begin{matrix}x_{1} & x_{2} & x_{3}\\x_{4} & x_{5} & x_{6}\\x_{7} & x_{8} & x_{9}\end{matrix}\right]
\end{equation}
then considering that 
\begin{align} 
	(A-\lambda_1)\matr v_1 & =0 \\
	(A-\lambda_2)\matr v_2 & =0 
\end{align}
hence
\begin{align} 
	\left[\begin{matrix}x_{1} - x_{2} - 1   \\x_{4} - x_{5} + 1\\x_{7} - x_{8}\end{matrix}\right] &= 0 \\
	\left[\begin{matrix}2 x_{1} + x_{3} + 2 \\2 x_{4} + x_{6}\\2 x_{7} + x_{9} + 1\end{matrix}\right] &= 0
\end{align}
by solving 
\begin{align}
	x_1-x_2    & =1   \\
	2x_1+x_3   & = -2 \\ \\
	x_4-x_5    & = -1 \\
	2x_4 + x_6 & = 0  \\ \\
	x_7-x_8    & = 0  \\
	2x_7+x_9   & = -1 
\end{align}
we have
\begin{equation}
	\left ( - \frac{x_{3}}{2} - 1, \quad - \frac{x_{3}}{2} - 2, \quad x_{3}, \quad - \frac{x_{6}}{2}, \quad - \frac{x_{6}}{2} + 1, \quad x_{6}, \quad - \frac{x_{9}}{2} - \frac{1}{2}, \quad - \frac{x_{9}}{2} - \frac{1}{2}, \quad x_{9}\right )
\end{equation}
hence we can express $\matr A$ as
\begin{equation}
	\left[\begin{matrix}- \frac{x_{3}}{2} - 1 & - \frac{x_{3}}{2} - 2 & x_{3}\\- \frac{x_{6}}{2} & - \frac{x_{6}}{2} + 1 & x_{6}\\- \frac{x_{9}}{2} - \frac{1}{2} & - \frac{x_{9}}{2} - \frac{1}{2} & x_{9}\end{matrix}\right]
\end{equation}
Since any values of $x_3,x_6$ and $x_9$ will meet the condition (48) and (49), \\
hence we can put $x_3=2, x_6=2$ and $x_9=2$, getting
\begin{equation}
	\matr A = \left[\begin{matrix}-2 & -3 & 2\\-1 & 0 & 2\\- \frac{3}{2} & - \frac{3}{2} & 2\end{matrix}\right]
\end{equation}
\begin{align}
	\matr A-\lambda \matr I  = & \left[\begin{matrix}- \lambda - 2 & -3 & 2\\-1 & - \lambda & 2\\- \frac{3}{2} & - \frac{3}{2} & - \lambda + 2\end{matrix}\right] \\
	|\matr A - \lambda \matr I|=& - \lambda \left(- \lambda - 2\right) \left(- \lambda + 2\right) - 3 \lambda \\
	=&-\lambda(\lambda-1)(\lambda+1)
\end{align}
hence with $|\matr A - \lambda \matr I|=0$, we can know that the third eigenvalue is $0$ \\ \\
For eigenvalue $0$ and eigenvectors $v_3$ for $\matr A$, we have
\begin{align}
    \left(A-0\matr I \right) v_3 &= 0 \\
    \left[\begin{matrix}-2 & -3 & 2\\-1 & 0 & 2\\- \frac{3}{2} & - \frac{3}{2} & 2\end{matrix}\right] v_3 &= 0 
\end{align}
hence by solving linear equation
\begin{equation}
    \begin{bmatrix}[ccc|c]
    -2 & -3 & 2 & 0\\-1 & 0 & 2 & 0\\- \frac{3}{2} & - \frac{3}{2} & 2 & 0
    \end{bmatrix}
\end{equation}
we have 
\begin{equation}
    v_3 = k \begin{bmatrix}
    2 \\ -\frac 2 3 \\ 1
    \end{bmatrix}
\end{equation}
Taking $v_3 = \begin{bmatrix}
    6 \\ -2 \\ 3
    \end{bmatrix} $  for simplicity\\
Therefore, we can digaonize $\matr A$ as
\begin{align}
    \matr V =& \begin{bmatrix}v_1 v_3 v_2\end{bmatrix} \\
    =&\left[\begin{matrix}2 & 6 & -1\\0 & -2 & 1\\1 & 3 & 0\end{matrix}\right] \\
    \matr V ^ -1=&a
\end{align}


% --------------------------------------------------------------
%     You don't have to mess with anything below this line.
% --------------------------------------------------------------
\end{document}