% --------------------------------------------------------------
% This is all preamble stuff that you don't have to worry about.
% Head down to where it says "Start here"
% --------------------------------------------------------------
 
\documentclass[12pt]{article}
 
    \usepackage[margin=1in]{geometry} 
    \usepackage{amsmath,amsthm}
     
    \usepackage{graphicx}
    
    \usepackage{tikz}
    
    \usepackage{amssymb}
    
    \usepackage{listings}
    \usepackage{color}
     
    \definecolor{codegreen}{rgb}{0,0.6,0}
    \definecolor{codegray}{rgb}{0.5,0.5,0.5}
    \definecolor{codepurple}{rgb}{0.58,0,0.82}
    \definecolor{backcolour}{rgb}{0.95,0.95,0.92}
     
    \lstdefinestyle{mystyle}{
        backgroundcolor=\color{backcolour},   
        commentstyle=\color{codegreen},
        keywordstyle=\color{magenta},
        numberstyle=\tiny\color{codegray},
        stringstyle=\color{codepurple},
        basicstyle=\footnotesize,
        breakatwhitespace=false,         
        breaklines=true,                 
        captionpos=b,                    
        keepspaces=true,                 
        numbers=left,                    
        numbersep=5pt,                  
        showspaces=false,                
        showstringspaces=false,
        showtabs=false,                  
        tabsize=2
    }
     
    \lstset{style=mystyle}
    
    \let\oldemptyset\emptyset
    \let\emptyset\varnothing
    
    \newcommand{\N}{\mathbb{N}}
    \newcommand{\Z}{\mathbb{Z}}
     
    \newenvironment{theorem}[2][Theorem]{\begin{trivlist}
    \item[\hskip \labelsep {\bfseries #1}\hskip \labelsep {\bfseries #2.}]}{\end{trivlist}}
    \newenvironment{lemma}[2][Lemma]{\begin{trivlist}
    \item[\hskip \labelsep {\bfseries #1}\hskip \labelsep {\bfseries #2.}]}{\end{trivlist}}
    \newenvironment{exercise}[2][Exercise]{\begin{trivlist}
    \item[\hskip \labelsep {\bfseries #1}\hskip \labelsep {\bfseries #2.}]}{\end{trivlist}}
    \newenvironment{reflection}[2][Reflection]{\begin{trivlist}
    \item[\hskip \labelsep {\bfseries #1}\hskip \labelsep {\bfseries #2.}]}{\end{trivlist}}
    \newenvironment{proposition}[2][Proposition]{\begin{trivlist}
    \item[\hskip \labelsep {\bfseries #1}\hskip \labelsep {\bfseries #2.}]}{\end{trivlist}}
    \newenvironment{corollary}[2][Corollary]{\begin{trivlist}
    \item[\hskip \labelsep {\bfseries #1}\hskip \labelsep {\bfseries #2.}]}{\end{trivlist}}
    \newenvironment{question}[2][Q]{\begin{trivlist}
    \item[\hskip \labelsep {\bfseries #1}\hskip \labelsep {\bfseries #2.}]}{\end{trivlist}}
    \newenvironment{answer}[2][A]{\begin{trivlist}
    \item[\hskip \labelsep {\bfseries #1}\hskip \labelsep {\bfseries #2.}]}{\end{trivlist}}
    
     
\begin{document}

% --------------------------------------------------------------
%                         Start here
% --------------------------------------------------------------



\title{Assignment 5}%replace X with the appropriate number
\author{Michael Lee\\ %replace with your name
	CCST9017 - Hidden Order in Daily Life: A Mathematical Perspective \\
	University Number 3035569110 \\
	Tutorial Group 009
} %if necessary, replace with your course title

\maketitle

\begin{question}{1a}
	Model this as a simple game.
\end{question}
\begin{answer}{1a}
	This game can be modeled as following
\end{answer}
\begin{equation}
	[13;10,5,2,5]
\end{equation}
with voter A, B, C, D as 1, 2, 3, 4 respectively

\begin{question}{1b}
	Does the simple game defined in a) has property M ? Explain your answer.
\end{question}
\begin{answer}{1b}
	Such game has property M
\end{answer}
$\because$ For $\{1,2\}$, which is a winning collation, all $\{1,2\}\subset S$ ($\{1,2,3,4\},\{1,2,4\},\{1,2,3\}$) is a winning collation \\
and \\
For $\{1,4\}$, which is a winning collation, all $\{1,4\}\subset S$ ($\{1,2,3,4\},\{1,2,4\},\{1,3,4\}$) is a winning collation \\
and \\
For $\{1,3\}$, which is a losing collation, all $S\subset\{1,3\}$ ($\{1\},\{3\}$) is a losing collation \\
$\therefore$ by the definition of property M, such game has property M \\

\begin{question}{1c}
	Are B and D symmetric players for the simple game defined in a)? Explain your answer.
\end{question}
\begin{answer}{1c}
	B and D are symmetric players because they have equal values of $5$.
\end{answer}

\begin{question}{1d}
	Find the Shapley-Shubikindex of each voter.
\end{question}
\begin{answer}{1d}

\end{answer}
There are totally $4!=24$ colations \\
For player 1, \\
player 1 is pivotal when $N^i_\sigma$ is $\{2\},\{4\},\{2,3\},\{2,4\}$ and $\{3,4\}$ \\
Hence number of permutation that player 1 is pivotal is given by
\begin{equation}
	2 \times 1! \times 2!+3 \times 2! \times 1!=10
\end{equation}
Hence $\textrm{SSI}(1)=\frac{10}{24}=\frac 5 {12}$
For player 2, \\
player 2 is pivotal when $N^i_\sigma$ is $\{1\},\{1,3\}$ and $\{3,4\}$ \\
Hence number of permutation that player 2 is pivotal is given by
\begin{equation}
	1 \times 1! \times 2! + 2 \times 2! \times 1! = 6
\end{equation}
Since player B and D are symmetric players, both $\textrm{SSI}(2)$ and $\textrm{SSI}(4)$ is $\frac 6 {24} = \frac 1 4$ \\
For player 3, \\
player 3 is pivotal when $N^i_\sigma$ is $\{2,4\}$\\
Hence number of permutation that player 3 is pivotal is given by
\begin{equation}
	1 \times 1! \times 2! = 2
\end{equation}
Hence $\textrm{SSI}(3)=\frac{2}{24}=\frac 1 {12}$

\begin{question}{2a}
	Show that a small state (non-permanent member) is a pivotal player for a permutation of the 15 states when seven small states come last in the permutation.
\end{question}
\begin{answer}{2a}
\end{answer}
When 7 small states come last, \\
The first $15-7=8$ players must includes all five big states, hence there are no veto situations. \\
And the 9th player, which is the 1st one of the 7 small states, becomes a pivotal player because her she has the choice of wheather become the 9th player to a yes vote, which passes the motion.

\begin{question}{2b}
	Show that a permutation of the 15 states in which a small state (non-permanent member)  is a pivotal player can occur only if seven small states come last in the permutation.
\end{question}
\begin{answer}{2b}
\end{answer}
For a small state $i$ to be a pivotal member in $\sigma$,\\
The 5 big state must be  in front of $i$, since passage requires the agreement of 5 big states. \\
Also, to statisfy the condition of passage of $9$ votes, another 3 small state must be in front of $i$ as well, so that $i$ becomes a pivotal player as mentioned in (a). \\
Therefore, the remaining last $15-5-3=7$ members must be small states.
Hence a small state (non-permanent member) is a pivotal player can occur only if seven small states come last in the permutation.

\begin{question}{2c}
	Show that the SSI of any nonpermanent member is 0.00186.
\end{question}
\begin{answer}{2c}
\end{answer}
From (b), a small state (non-permanent member) is a pivotal player can occur only if seven small states come last in the permutation. \\
Therefore, the first 8 players of such permutation $\sigma$ must include 5 big states and 3 of 10 small states. \\
And the last 7 players of $\sigma$ must include 7 of 10 small states. \\
Hence number of permutations statisfying such requirements is given by
\begin{equation}
	8! \times {}_{10} \mathrm P_7 = 2.4385536 \times 10^{10}
\end{equation}
Hence such SSI is given by $2.4385536 \times 10^{10} \div 15! = 0.0186$

\begin{question}{2d}
	Find the SSI of any permanent member.
\end{question}
\begin{answer}{2c}
	Such SSI is given by $\frac{1-0.0186\times 10}5=0.163$
\end{answer}

% --------------------------------------------------------------
%     You don't have to mess with anything below this line.
% --------------------------------------------------------------

\end{document}