% --------------------------------------------------------------
% This is all preamble stuff that you don't have to worry about.
% Head down to where it says "Start here"
% --------------------------------------------------------------
 

\documentclass[12pt]{article}
    \usepackage[margin=1in]{geometry} 
	\usepackage{amsmath,amsthm,amssymb,bm}
	\usepackage{mathtools}
	\makeatletter
\renewcommand*\env@matrix[1][*\c@MaxMatrixCols c]{%
  \hskip -\arraycolsep
  \let\@ifnextchar\new@ifnextchar
  \array{#1}}
\makeatother
     
    \newcommand{\N}{\mathbb{N}}
    \newcommand{\Z}{\mathbb{Z}}
    \newcommand{\matr}[1]{\bm{#1}}
    \newenvironment{theorem}[2][Theorem]{\begin{trivlist}
    \item[\hskip \labelsep {\bfseries #1}\hskip \labelsep {\bfseries #2.}]}{\end{trivlist}}
    \newenvironment{lemma}[2][Lemma]{\begin{trivlist}
    \item[\hskip \labelsep {\bfseries #1}\hskip \labelsep {\bfseries #2.}]}{\end{trivlist}}
    \newenvironment{exercise}[2][Exercise]{\begin{trivlist}
    \item[\hskip \labelsep {\bfseries #1}\hskip \labelsep {\bfseries #2.}]}{\end{trivlist}}
    \newenvironment{reflection}[2][Reflection]{\begin{trivlist}
    \item[\hskip \labelsep {\bfseries #1}\hskip \labelsep {\bfseries #2.}]}{\end{trivlist}}
    \newenvironment{proposition}[2][Proposition]{\begin{trivlist}
    \item[\hskip \labelsep {\bfseries #1}\hskip \labelsep {\bfseries #2.}]}{\end{trivlist}}
    \newenvironment{corollary}[2][Corollary]{\begin{trivlist}
    \item[\hskip \labelsep {\bfseries #1}\hskip \labelsep {\bfseries #2.}]}{\end{trivlist}}
    \newenvironment{question}[2][Question]{\begin{trivlist}
    \item[\hskip \labelsep {\bfseries #1}\hskip \labelsep {\bfseries #2.}]}{\end{trivlist}}
    \newenvironment{answer}[2][Answer]{\begin{trivlist}
    \item[\hskip \labelsep {\bfseries #1}\hskip \labelsep {\bfseries #2.}]}{\end{trivlist}}
     \DeclarePairedDelimiterX\set[1]\lbrace\rbrace{\def\given{\;\delimsize\vert\;}#1}
     
    \begin{document}
    
% --------------------------------------------------------------
%                         Start here
% --------------------------------------------------------------
    
%\renewcommand{\qedsymbol}{\filledbox}
    
\title{Assignment 3}%replace X with the appropriate number
\author{Michael Lee\\ %replace with your name
	MATH1853 \\
	University Number 3035569110 
} %if necessary, replace with your course title
    
    
\maketitle
\begin{answer}{Q. 1a}
\hfill \break
Let $z=e^{i\theta}$, then we have
\begin{align}
    e^{6i\theta} &= 1 \\
    6i\theta &= 0,2\pi i, 4\pi i, 6\pi i,8\pi i, 10\pi i, 12\pi i, 14\pi i \hdots \\
    \theta &=0,\pi /3,2\pi /3 ,\pi, 4\pi /3, 5\pi /3, 2\pi ,7\pi /3 \\
    z&= 1,-1,\frac {1+\sqrt{3}} 2,\frac {-1+\sqrt{3}} 2,\frac {-1+-\sqrt{3}} 2,\frac {1-\sqrt{3}} 2 \\
    \because z &\neq -1 \\
    \therefore S&=\left\{1,\frac {1+\sqrt{3}} 2,\frac {-1+\sqrt{3}} 2,\frac {-1+-\sqrt{ 1`3}} 2,\frac {1-\sqrt{3}} 2\right\}
\end{align}
\end{answer}
\begin{answer}{Q. 1b}
    $$|S|=5$$
\end{answer}
\begin{answer}{Q. 2}
    \begin{align}
        \left(z+1\right)^4 &= 5 \\
        \left(z+1\right)^2 &= 5^{1/2} , -5^{1/2} \\
        z+1&=5^{1/4}, -5^{1/4},5^{1/4}i, -5^{1/4}i \\
        z&=5^{1/4}-1, -5^{1/4}-1,5^{1/4}i-1, -5^{1/4}i-1 
    \end{align}
\end{answer}
\begin{answer}{Q. 3}
    \begin{align}
        & \left(-i\right)^{5/2} \\
        =& e^{5\pi /2 i}\\
        =& \cos {5\pi /2} + i\sin {5\pi /2} \\
        =& \cos {\pi /2} + i\sin {\pi /2} \\
        =& 
    \end{align}
\end{answer}
% --------------------------------------------------------------
%     You don't have to mess with anything below this line.
% --------------------------------------------------------------
\end{document}